\documentclass{article}

\usepackage[T1]{fontenc}
\usepackage[utf8]{inputenc}
\usepackage{listings}
\usepackage{courier}
\usepackage{enumitem}

\lstset{
    basicstyle=\ttfamily\small,
    breaklines=true,
}

\begin{document}

\section{Customizing SQLSmith for SQL Query Generation}

\subsection{SQLSmith}
SQLSmith is an open-source tool for randomized generation of SQL queries,
designed originally for database fuzz testing.

SQLSmith is implemented as a grammar-driven random generator.
It relies on three main components:
\begin{enumerate}
    \item \textbf{\texttt{grammar}}: grammar production, defines the syntactic structure of queries
          (SELECT, DELETE, MERGE, FROM, JOIN, etc.);
    \item \textbf{\texttt{expr}}: value expressions production, generates expression including values,
          predicates, comparisons, functions, and aggregations;
    \item \textbf{\texttt{relmodel}}: models relations, columns, and aliases
          such as \texttt{ref\_0}, \texttt{subq\_0}, and \texttt{c0} which SQLSmith
          uses internally during generation.
\end{enumerate}


\subsection{Motivation for Customization}
The goal is to adapt SQLSmith to generate a dataset of
\emph{realistic and structurally controlled} SQL queries, suitable for being used in ProvSQL and LLM fine-tuning.
This required constraining or removing the following behaviours:
\begin{itemize}
    \item automatically generated aliases (\texttt{ref\_0}, \texttt{c12}, etc.);
    \item nonstandard system functions (distributed functions, probability functions etc.);
    \item \texttt{TABLESAMPLE}, \texttt{LATERAL}, and CTEs;
    \item nested aggregates (e.g., \texttt{SUM (MIN (x))});
\end{itemize}

Three source files were modified: \texttt{grammar.cc}, \texttt{expr.cc}, \texttt{postgres.cc} and
\texttt{relmodel.hh}.

\subsection{Modifications to \texttt{grammar.cc}}
The file \texttt{grammar.cc} controls the high-level structure of SQL queries.
The following customizations were applied:
\begin{itemize}
    \item disabling constructs that introduce uncontrolled complexity:
    \begin{itemize}
        \item \texttt{table\_subquery};
        \item \texttt{lateral\_subquery};
        \item \texttt{table\_sample};
        \item subqueries generated via \texttt{WITH} (CTEs).
    \end{itemize}
    \item rewriting the \texttt{table\_ref::factory} method to produce
          only real base tables or joins between base tables;
    \item simplifying the \texttt{SELECT} list by removing automatically generated
          derived column aliases (\texttt{c0}, \texttt{ref0}, \ldots);
    \item enforcing the use of only \texttt{SELECT} statements (disabling
          \texttt{DELETE}, \texttt{UPDATE}, \texttt{MERGE}, etc.);
    \item allowed only inner joins by modifying the \texttt{join\_cond::factory}
          method to always set the join type to \texttt{inner}.
\end{itemize}


\subsection{Modifications to \texttt{expr.cc}}
The file \texttt{expr.cc} defines how expressions, predicates, and function calls
are generated. Some modifications were applied, as:
\begin{itemize}
    \item \textbf{Disabling subqueries in expressions}
          (\texttt{atomic\_subselect});
    \item \textbf{Disabling window functions}
          (\texttt{window\_function});
    \item \textbf{Avoiding nested aggregate functions} by modifying the
          \texttt{funcall} constructor:
\begin{lstlisting}

if (agg && dynamic_cast<funcall*>(p->pprod)) {
    fail("nested aggregate not allowed");
}
\end{lstlisting}

    \item \textbf{Removing complex or system-defined functions} from the generative process.
    \item \textbf{Removing exists predicates} from the generative process because it is not supported in ProvSQL.
    \item \textbf{Modifying the set of available operators} to use only basic operators in comparisons.
    \begin{lstlisting}
        bool name_ok =
        name == "="  ||
        name == "<"  ||
        name == ">"  ||
        name == "<=" ||
        name == ">=";
    \end{lstlisting}
\end{itemize}

\subsection{Modifications to \texttt{postgres.cc}}
The file \texttt{postgres.cc} defines PostgreSQL-specific functions and types.
The following changes were applied:
\begin{itemize}
    \item removing PostgreSQL aggregates and routines from the generative process,
          such as \texttt{pg\_catalog.random()}, \texttt{pg\_catalog.setseed()}.
\end{itemize}
\subsection{Modifications to \texttt{relmodel.hh}}
The file \texttt{relmodel.hh} defines the representation of tables,
relations, and aliasing.
SQLSmith generates internal aliases such as \texttt{ref\_0}, \texttt{ref\_1},
and derived column names such as \texttt{c0}, \texttt{c1}.

The following changes were applied:
\begin{itemize}
    \item aliases in the table naming and routine naming are disabled, using the table name directly.

\end{itemize}

This transforms outputs like:

\begin{lstlisting}
FROM public.partsupp
FROM system.partsupp 
SELECT pg_catalog.min()
\end{lstlisting}

into:

\begin{lstlisting}
FROM partsupp
FROM partsupp
SELECT MIN()
\end{lstlisting}


\subsection{Resulting Behaviour}
After applying these modifications, SQLSmith produces:
\begin{itemize}
    \item only \texttt{SELECT} queries (no CTEs, no MERGE/UPDATE/DELETE);
    \item joins between real base tables only;
    \item simple and readable \texttt{SELECT} lists without synthetic aliases;
    \item aggregate functions without nesting;
\end{itemize}


\end{document}